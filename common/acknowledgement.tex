
Работы, описанные в диссертации были поддержаны рядом российских и международных грантов и стипендий, в том числе, грантами РНФ 18—74—0006, 14-24-00031, 19-74-30003, РФФИ 20-34-70039, 12-04-31942, стипендией сотрудничества России-США в области биомедицинских наук, стипендией Национальной медицинской библиотеки США, Немецким научно-исследовательским обществом, проект SFB 569 A11. Работа выполнена с использованием оборудования Центра коллективного пользования сверхвысокопроизводительными вычислительными ресурсами МГУ имени М.В. Ломоносова, кластера Biowulf Национальных институтов здоровья (США), кластера Aldan университета г. Ульм (Германия).

Автор выражает благодарности своим научным руководителям и консультантам, под руководством которых автору посчастливилось работать, М.П.Кирпичникову, А.Р.Хохлову, А.Р. Панченко, Д. Ландсману, П.Г. Халатуру, В.А. Иванову, всем соавторам своих научных работ и коллегам за плодотворное сотрудничество, в особенности, В.Б. Журкину, В.М. Студитскому, К.Ву, Х. Жао, А. Гончаренко, Г.А. Армееву, И. Драйзену, Е.-К. Шиллингер, О.С. Соколовой, А.В. Феофанову, Н.В. Малюченко, Е. Бондаренко, М. Валиевой, А. Любителеву и многим другим, коллективам кафедры физики полимеров и кристаллов физического факультета МГУ, кафедры биоинженерии биологического факультета МГУ, Национального Центра Биотехнологической Информации Национальных Институтов Здоровья за продуктивную рабочую атмосферу и обсуждение работы.

Автор выражает благодарность своей семье за поддержку, без которой написание этой работы не было бы возможным, и А.Д. Шайтану за помощь с версткой текста.
